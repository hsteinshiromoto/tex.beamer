% !TEX encoding = UTF-8 Unicode
% !TEX TS-program = xelatex
% !TEX root = ../main.tex

% ---
% AMS Packages
% ---
\usepackage{amsmath,amsfonts,amssymb} % AMS fonts and symbols package
\usepackage{mathrsfs}

% ---
% Typeface Settings
% ---
\usefonttheme{professionalfonts} %Necessary to use with Euler

 % Load BEFORE mathspec AND fontspec packages
\usepackage[small,euler-digits,euler-hat-accent]{eulervm}

\usepackage{mathspec}
		
\usepackage{fontspec}
	\defaultfontfeatures{Mapping=tex-text,Ligatures={TeX,Common},
                       Scale=MatchUppercase,Color=textcolor}
	
	\setsansfont[
        Extension      = .otf,
        UprightFont    = *-regular,
        BoldFont       = *-bold,
        ItalicFont     = *-italic,
        BoldItalicFont = *-bolditalic]
				{texgyreheros}

% ---
% Language Settings
% ---

% Needs to be loaded locally and after the fontspec package
\usepackage{polyglossia}
\setdefaultlanguage[variant=british]{english}
\setotherlanguages{french,portuges,italian}

\usepackage{csquotes}

% ---
% Biblatex Package
% ---

% To be loaded locally after the language package
\usepackage[natbib=true,%
backend=biber,%
style=alphabetic,%
sorting=anyvt,%
firstinits=true,%
doi=true,%
isbn=false,%
url=false,%
backref=true,%
texencoding=utf8,%
bibencoding=utf8]{biblatex}
\IfFileExists{./Library.bib}{

 \addbibresource{./Library}
 
}{

  \addbibresource[location=remote]{https://gitlab.com/hsteinshiromoto/bibliography/raw/master/Library.bib}  

}

% ---
% Graphic settings
% ---
\usepackage{graphicx,xcolor,epstopdf}
\usepackage{color}
\DeclareGraphicsRule{.tif}{png}{.png}{`convert #1 `dirname #1`/`basename #1 .tif`.png}

% Color definitions
\definecolor{canary}{HTML}{FCF0AD}

\definecolor{gindigo}{HTML}{2196F3}

%Blue
\definecolor{gblue500}{HTML}{2196F3}
\definecolor{gblue50}{HTML}{E3F2FD}
\definecolor{gblue100}{HTML}{BBDEFB}
\definecolor{gblue200}{HTML}{90CAF9}
\definecolor{gblue300}{HTML}{64B5F6}
\definecolor{gblue400}{HTML}{42A5F5}
\definecolor{gblue600}{HTML}{1E88E5}
\definecolor{gblue700}{HTML}{1976D2}
\definecolor{gblue800}{HTML}{1565C0}
\definecolor{gbluetext}{HTML}{1565C0}
\definecolor{gblue900}{HTML}{0D47A1}

%Green
\definecolor{ggreen500}{HTML}{4CAF50}
\definecolor{ggreen50}{HTML}{E8F5E9}
\definecolor{ggreen100}{HTML}{C8E6C9}
\definecolor{ggreen200}{HTML}{A5D6A7}
\definecolor{ggreen300}{HTML}{81C784}
\definecolor{ggreen400}{HTML}{66BB6A}
\definecolor{ggreen500}{HTML}{4CAF50}
\definecolor{ggreen600}{HTML}{43A047}
\definecolor{ggreen700}{HTML}{388E3C}
\definecolor{ggreen800}{HTML}{2E7D32}
\definecolor{ggreen900}{HTML}{1B5E20}

%Red
\definecolor{ggrey500}{HTML}{9E9E9E}
\definecolor{ggrey50}{HTML}{FAFAFA}
\definecolor{ggrey100}{HTML}{F5F5F5}
\definecolor{ggrey200}{HTML}{EEEEEE}
\definecolor{ggrey300}{HTML}{E0E0E0}
\definecolor{ggrey400}{HTML}{BDBDBD}
\definecolor{ggrey500}{HTML}{9E9E9E}
\definecolor{ggrey600}{HTML}{757575}
\definecolor{ggrey700}{HTML}{616161}
\definecolor{ggrey800}{HTML}{424242}
\definecolor{ggrey900}{HTML}{212121}

%Grey
\definecolor{gred500}{HTML}{F44336}
\definecolor{gred50}{HTML}{FFEBEE}
\definecolor{gred100}{HTML}{FFCDD2}
\definecolor{gred200}{HTML}{EF9A9A}
\definecolor{gred300}{HTML}{E57373}
\definecolor{gred400}{HTML}{EF5350}
\definecolor{gred500}{HTML}{F44336}
\definecolor{gred600}{HTML}{E53935}
\definecolor{gred700}{HTML}{D32F2F}
\definecolor{gred800}{HTML}{C62828}
\definecolor{gredtext}{HTML}{C62828}
\definecolor{gred900}{HTML}{B71C1C}

%Yellow
\definecolor{gyellow500}{HTML}{FFEB3B}
\definecolor{gyellow50}{HTML}{FFFDE7}
\definecolor{gyellow100}{HTML}{FFF9C4}
\definecolor{gyellow200}{HTML}{FFF59D}
\definecolor{gyellow300}{HTML}{FFF176}
\definecolor{gyellow400}{HTML}{FFEE58}
\definecolor{gyellow600}{HTML}{FDD835}
\definecolor{gyellow700}{HTML}{FBC02D}
\definecolor{gyellow800}{HTML}{F9A825}
\definecolor{gyellow900}{HTML}{F57F17}

% Orange
\definecolor{gorange500}{HTML}{FF9800}

% Postit color
\definecolor{postitcolor}{HTML}{FCF0AD}

% Todo: Review the need of graph the packages below
\usepackage{pdfrender}
\usepackage[beamer]{hf-tikz}
\usepackage{tikz}
\usetikzlibrary{arrows,shapes,shadings}
\usetikzlibrary{calc}

\usepackage{multimedia,animate, colortbl}

\usepackage{tcolorbox} 
\tcbuselibrary{skins,theorems}
\usepackage{varwidth} % To change blocks width

% ---
% Other Packages
% ---
% Separates counting from actual presentation from appendix
\usepackage{appendixnumberbeamer}

% Reduce left margin
\usepackage{changepage}

% Conditional package
\usepackage{ifthen,forloop}

\usepackage{multicol}
\setlength{\columnseprule}{0.4pt}

% Dummy text generator
\usepackage{blindtext}

\usepackage{verbatim}
% ---
% Water mark 
% ---

% Todo: review the need of the commented code below

% different for odd and even pages
%\usepackage{draftwatermark}
%
%\makeatletter
%\renewcommand\sc@wm@print[1]{% redefine positioning of mark (-1in to 330pt)
% \if@sc@wm@stamp
%          \setbox\@tempboxa\vbox to \z@{%
%          \ifodd\c@page
%            \vskip 2em  \moveright 8em
%           \else
%            \vskip 2em  \moveright 12.5em
%           \fi \vbox{%
%              \hbox to \z@{%
%                #1\hss}}\vss}
%          \dp\@tempboxa\z@
%          \box\@tempboxa
%          \fi}
%\makeatother
%
%\makeatletter
%\SetWatermarkText{%
%    \ifodd\c@page
%        \begin{tikzpicture}\node[opacity=.1]		{\includegraphics[angle=315,scale=0.0625]{./imgs/MIT}};				\end{tikzpicture}
%    \else
%    	\begin{tikzpicture}\node[opacity=.1]		{\includegraphics[angle=315,scale=1]{./imgs/Usyd_new_logo}};
%   		\end{tikzpicture}
%    \fi
%}
%\makeatother
%
%\setbeamertemplate{background canvas}{}

% ---
% Hyperref Settings
% ---
\usepackage{hyperref}% backref linktocpage pagebackref

\hypersetup{
colorlinks=true, linktocpage=true, pdfstartpage=1, pdfstartview=FitV,
% Uncomment the line below if you want to have black links (e.g. for printing black and white)
colorlinks=false, linktocpage=false, pdfborder={0 0 0}, 
pdfstartpage=1, pdfstartview=FitV, 
pdftitle={Presentation},
pdfauthor={Humberto STEIN SHIROMOTO},
pdfsubject={},
pdfkeywords={},
pdfcreator={Sublime Text 3},
pdfproducer={}
}

% Todo: Review this beamer color commands... maybe move to other parts?
\setbeamercolor{bibliography entry author}{fg=ggreentext}
\setbeamercolor{bibliography entry title}{fg=black}
\setbeamercolor{bibliography entry location}{fg=gredtext}
\setbeamercolor{bibliography entry Publisher}{fg=gredtext}
\setbeamercolor{bibliography entry note}{fg=black}
\setbeamercolor{bibliography item}{fg=gblue700}

% ---
% Itemize and enumerate style
% ---
\setbeamertemplate{itemize item}[square]
\setbeamercolor{itemize item}{fg=gblue700}

 % if you want a circle
\setbeamertemplate{itemize subitem}[circle]
\setbeamercolor{itemize subitem}{fg=gblue700}

\setbeamertemplate{enumerate item}[square]
\setbeamercolor{enumerate item}{fg=gblue700,bg=gblue700}

\setbeamercolor{local structure}{fg=gblue700}

% Removes shadow from title page box
\setbeamertemplate{title page}[default][colsep=-4bp,rounded=true]

% ---
% Frametitle
% ---
% Removes shadow from frametitle
\setbeamertemplate{frametitle}[default][colsep=-4bp,rounded=false,shadow=false]

\setbeamercolor{frametitle}{fg=white,bg=gblue700}

% ---
% Headline
% ---
% Todo: review the need of the commented code below

%\setbeamerfont{headline}{series=\bfseries}
\setbeamercolor{section in head/foot}{fg=white, bg=black}
% \setbeamercolor{upper separation line head}{bg=white}
\setbeamertemplate{section in head/foot shaded}%
                  {\color{white!100!black}\insertsectionhead}

% \makeatletter
% \setbeamertemplate{headline}
% {%
%   \pgfuseshading{beamer@barshade}%
%   \ifbeamer@sb@subsection%
%     \vskip-9.75ex%
%   \else%
%     \vskip-7ex%
%   \begin{beamercolorbox}[ignorebg,ht=
%   \fi%2.25ex,dp=3.75ex]{section in head/foot}
%     \insertnavigation{\paperwidth}
%   \end{beamercolorbox}%
%   \ifbeamer@sb@subsection%
%     \begin{beamercolorbox}[ignorebg,ht=2.125ex,dp=1.125ex,%
%       leftskip=.3cm,rightskip=.3cm plus1fil]{subsection in head/foot}
%       \usebeamerfont{subsection in head/foot}\insertsubsectionhead
%     \end{beamercolorbox}%
%   \fi%
%  \begin{beamercolorbox}[colsep=1.5pt,ht=0.75ex]{upper separation line head}
%  \end{beamercolorbox}
% }%
% \makeatother


% ---
% Footline
% ---
\setbeamercolor{footline}{fg=white,bg=black}

\makeatletter
\defbeamertemplate*{footline}{my footline}
{
   \ifnum \insertpagenumber=1
      \leavevmode%
      \hbox{%
      \begin{beamercolorbox}[wd=\paperwidth,ht=2.25ex,dp=1ex,center]{}%
        % empty environment to raise height
      \end{beamercolorbox}}%
      \vskip0pt%
    \else
  \leavevmode%
  \hbox{%
  \begin{beamercolorbox}[wd=.25\paperwidth,ht=2.25ex,dp=1ex,left]{footline}%
    \usebeamerfont{author in head/foot}\hspace{2em}\insertshortauthor
  \end{beamercolorbox}%
  \begin{beamercolorbox}[wd=.75\paperwidth,ht=2.25ex,dp=1ex,right]{footline}%
    \usebeamerfont{title in head/foot}\insertshorttitle\hfill (\insertframenumber{} | \inserttotalframenumber) \hspace{2em}
  \end{beamercolorbox}
  }%
  \vskip0pt%
  \fi
}
\makeatother

% ---
% Section titlepage
% ---

% Todo: Add a \if condition
\ifsectitlepage
  \AtBeginSection[]{%
      {
      \setbeamercolor{upper separation line head}{bg=gblue700}
      \begin{frame}
      \begin{tikzpicture}[remember picture,overlay]
      \fill[gblue700]
          ([yshift=0pt]current page.west) rectangle ([yshift=-\headheight] current page.north east);
      \node[anchor=center]
          at ([yshift=10pt,xshift=0pt]current page.center) 
          {\parbox[t]{\textwidth}{\centering%
              \usebeamerfont{author}\textcolor{white}{%
                      \textpdfrender{
                          TextRenderingMode=FillStroke,
                          FillColor=white,
                          LineWidth=.1ex,
                      }{\insertsection}}}};
      \end{tikzpicture}
      \end{frame}}
      \addtocounter{framenumber}{-1}
  }
\fi

% ---
% Table of contents
% ---

%Changing numbering in TOC
\setbeamertemplate{sections/subsections in toc}[square]
% \setbeamertemplate{section in toc}{\inserttocsection}
\setbeamerfont{section in toc}{series=\bfseries}

% TOC before each section
% \AtBeginSection[]{
% 	\begin{frame}<beamer>
% 		%\frametitle{Overview}
% 		\tableofcontents[currentsection,hideallsubsections,subsubsectionstyle=hide]
% 	\end{frame}
% 	\addtocounter{framenumber}{-1} 
% }

